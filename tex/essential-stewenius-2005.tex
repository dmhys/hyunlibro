\documentclass[tikz,border=10pt]{standalone}
\usepackage{tikz}
\usepackage{colortbl}
\usetikzlibrary{matrix,positioning}


\begin{document}

\tikzset{ 
table/.style={
  matrix of math nodes,
  row sep=-\pgflinewidth,
  column sep=-\pgflinewidth,
  nodes={align=center},
  text width=20pt,
  text depth=7pt,
  text height=15pt,
  nodes in empty cells,
  }
}

\pgfdeclarelayer{bg}    % declare background layer
\pgfsetlayers{bg,main}  % set the order of the layers (main is the standard layer)

\begin{tikzpicture}
\matrix (m) [table] {
A & x^3 & x^2y & x^2z & xy^2 & xyz & xz^2 & y^3 & y^2z & yz^2 & z^3 & x^2 & xy & xz & y^2 & yz & z^2 & x & y & z & 1 \\
(a) & 1 & & & & & & & & & & \cdot & \cdot & \cdot & \cdot & \cdot & \cdot & \cdot & \cdot & \cdot & \cdot \\
(b) & & 1 & & & & & & & & & \cdot & \cdot & \cdot & \cdot & \cdot & \cdot & \cdot & \cdot & \cdot & \cdot \\
(c) & & & 1 & & & & & & & & \cdot & \cdot & \cdot & \cdot & \cdot & \cdot & \cdot & \cdot & \cdot & \cdot \\
(d) & & & & 1 & & & & & & & \cdot & \cdot & \cdot & \cdot & \cdot & \cdot & \cdot & \cdot & \cdot & \cdot \\
(e) & & & & & 1 & & & & & & \cdot & \cdot & \cdot & \cdot & \cdot & \cdot & \cdot & \cdot & \cdot & \cdot \\
(f) & & & & & & 1 & & & & & \cdot & \cdot & \cdot & \cdot & \cdot & \cdot & \cdot & \cdot & \cdot & \cdot \\
(g) & & & & & & & 1 & & & & \cdot & \cdot & \cdot & \cdot & \cdot & \cdot & \cdot & \cdot & \cdot & \cdot \\
(h) & & & & & & & & 1 & & & \cdot & \cdot & \cdot & \cdot & \cdot & \cdot & \cdot & \cdot & \cdot & \cdot \\
(i) & & & & & & & & & 1 & & \cdot & \cdot & \cdot & \cdot & \cdot & \cdot & \cdot & \cdot & \cdot & \cdot \\
(j) & & & & & & & & & & 1 & \cdot & \cdot & \cdot & \cdot & \cdot & \cdot & \cdot & \cdot & \cdot & \cdot \\
};

\draw[black, thick] (m-1-1.south west) -- (m-1-21.south east);
\draw[black, thick] (m-1-1.north east) -- (m-11-1.south east);

\begin{pgfonlayer}{bg}
\fill[cyan!40] ([xshift=0.75pt, yshift=-0.75pt] m-2-2.north west) rectangle (m-2-2.south east);
\fill[cyan!40] ([xshift=0.75pt, yshift=-0.75pt] m-3-3.north west) rectangle (m-3-3.south east);
\fill[cyan!40] ([xshift=0.75pt, yshift=-0.75pt] m-4-4.north west) rectangle (m-4-4.south east);
\fill[cyan!40] ([xshift=0.75pt, yshift=-0.75pt] m-5-5.north west) rectangle (m-5-5.south east);
\fill[cyan!40] ([xshift=0.75pt, yshift=-0.75pt] m-6-6.north west) rectangle (m-6-6.south east);
\fill[cyan!40] ([xshift=0.75pt, yshift=-0.75pt] m-7-7.north west) rectangle (m-7-7.south east);
\fill[cyan!40] ([xshift=0.75pt, yshift=-0.75pt] m-8-8.north west) rectangle (m-8-8.south east);
\fill[cyan!40] ([xshift=0.75pt, yshift=-0.75pt] m-9-9.north west) rectangle (m-9-9.south east);
\fill[cyan!40] ([xshift=0.75pt, yshift=-0.75pt] m-10-10.north west) rectangle (m-10-10.south east);
\fill[cyan!40] ([xshift=0.75pt, yshift=-0.75pt] m-11-11.north west) rectangle (m-11-11.south east);

\fill[cyan!40] ([xshift=0.75pt, yshift=-0.75pt] m-2-12.north west) rectangle (m-11-21.south east);

\end{pgfonlayer}

% \begin{scope}[shorten >= 10pt,shorten <= 10pt]
% \draw[red, thick] (m-4-2.north west) -- (m-4-8.north east) -- (m-4-8.south east) -- (m-4-2.south west) -- cycle;
% \end{scope}

\end{tikzpicture}
\end{document}